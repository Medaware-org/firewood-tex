\section{Prologue}
\subsection{Customer Traits}
It is reasonable to assume that, due to the inherent nature of the products offered by \fw, its customers must
exhibit one common trait, namely \textbf{the need for a mobile, localised source of heat energy}.
This definition, despite not being necessarily flawed, ignores a very important parameter of the
energy source --- the (average) thermal power output.
There is, after all, a vast difference between a cigarette lighter, and an acetylene torch.
\subsection{Power Calculations}
To narrow down our customer base, let's run some primitive calculations to determine the power range we're striving to
operate within.
Consider a typical lighter from a competing company, \textit{BIC}. Said lighter is equipped with $4.5g$ of butane fuel.
This amount of butane allows the device to maintain a (more or less) constant power output throughout the duration of $30min$.
Upon combustion, one mole of the butane fuel releases a fixed amount of energy into the environment. This energy is defined
by the \textit{enthalpy of combustion} $\dots$
\begin{align*}
	\Delta H_{c\textit{(mol)}} &= -2.88 \cdot 10^3 \: kJ \cdot mol^{-1}
\end{align*}
Thus, the device is capable of outputting an average amount of thermal power equal to $\dots$
\begin{align*}
	M &= 58.124 \: g \cdot mol^{-1} \\
	n &= \frac{4.5 \: g}{58.124 \: g \cdot mol^{-1}} \\
	  &= 7.74 \cdot 10^{-2} \: mol \\
	\Delta H_c &= \Delta H_{c\textit{(mol)}} \cdot n \\
		   &= -2.23 \cdot 10^2 \: kJ \\
	\bar{P} = \frac{E}{t} &= \frac{2.23 \cdot 10^2 \: kJ \cdot 10^3}{30min \cdot 60s} \\
			&= \frac{2.23 \cdot 10^5}{1.7 \cdot 10^3} \\
			&= \du{123.8 \: W}
\end{align*}
\subsubsection{Continuous Form}
Assuming that the fuel consuption is not constant, the formul{\ae} for the instantaneous and average power outputs
shall be expressed in a continuous form:
\begin{align*}
	P(t) &= \frac{dE}{dt} \\
	\bar{P} &= \frac{1}{T} \cdot \int\limits_{0}^{T} P(t) \: dt
\end{align*}
