\section{Prologue}
\subsection{Customer Traits}
It is reasonable to assume that, due to the inherent nature of the products offered by \fw, its customers must
exhibit one common trait, namely \textbf{the need for a mobile, localised source of heat energy}.
This definition, despite not immediately coming across as inherently flawed, ignroes a very important parameter of the
energy source, this being the (average) thermal power output. \\[\baselineskip]
\subsection{Power Calculations}
Consider a typical \textit{BIC} lighter equipped with $4.5g$ of butane fuel. Said fuel amount is capable of maintaining
a (more or less) constant power output throughout the duration of $30min$.
The fuel source used in most lighters is \textit{butane}. Upon combustion, one mole of said fuel
releases a fixed amount of energy into the environment. This energy is defined by the \textit{enthalpy of combustion} $\dots$
\begin{align*}
	\Delta H_{c\textit{(mol)}} &= -2.88 \cdot 10^3 \: kJ \cdot mol^{-1}
\end{align*}
Given the molar mass of butane $\dots$
\begin{align*}
	M &= 58.124 \: g \cdot mol^{-1}
\end{align*}
A full tank of butane contains:
\begin{align*}
	n &= \frac{4.5 \: g}{58.124 \: g \cdot mol^{-1}} \\
	  &= 7.74 \cdot 10^{-2} \: mol \\
\end{align*}
And is thus capable of delivering a fixed amount of total energy:
\begin{align*}
	\Delta H_c &= \Delta H_{c\textit{(mol)}} \cdot n \\
		   &= -2.23 \cdot 10^2 \: kJ
\end{align*}
Spreading that energy output out over a $30$-minute burn time, the average power output equals $\dots$
\begin{align*}
	\bar{P} = \frac{E}{t} &= \frac{2.23 \cdot 10^2 \: kJ \cdot 10^3}{30min \cdot 60s} \\
			&= \frac{2.23 \cdot 10^5}{1.7 \cdot 10^3} \\
			&= 123.8 \: W
\end{align*}
Assuming that the fuel consuption is not contant, the power formula may be expressed in a continuous form:
\begin{align*}
	P(t) &= \frac{dE}{dt}
\end{align*}
Thus, the average pwoer output becomes $\dots$
\begin{align*}
	\bar{P} &= \frac{1}{T} \cdot \int\limits_{0}^{T} P(t) \: dt
\end{align*}
